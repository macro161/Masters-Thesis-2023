\documentclass{VUMIFPSmagistrinis}
\usepackage{algorithmicx}
\usepackage{algorithm}
\usepackage{algpseudocode}
\usepackage{amsfonts}
\usepackage{amsmath}
\usepackage{bm}
\usepackage{caption}
\usepackage{color}
\usepackage{float}
\usepackage{graphicx}
\usepackage{listings}
\usepackage{subfig}
\usepackage{wrapfig}



\university{Vilniaus universitetas}
\faculty{Matematikos ir informatikos fakultetas}
\department{Informatikos institutas}
\papertype{Magistro darbo planas}
\title{Elixir proceso priežiūros įtaka išskirstytų algoritmų teisingumui}
\titleineng{Impact of Elixir process supervision on the correctness of distributed algorithms}
\status{2 kurso 1 grupės studentas}

\author{Matas Savickis}

\supervisor{Karolis Petrauskas, Doc., Dr.}
\reviewer{}
\date{Vilnius – \the\year}

\bibliography{bibliografija}

\begin{document}
\pagenumbering{arabic}

\maketitle

\tableofcontents



	\sectionnonum{Įvadas}
		Šiais laikais kai kurios programų sistemos yra išskirstytos \cite{mcr}. 
		Tokios sistemos, kaip sufleruoja pavadinimas, yra kuriamos išskirstant skaičiavimo mazgus į atskiras, savarankiškas dalis \cite{coulouris2005distributed}.
		Ne kaip monolitinės sistemos, išskirstytos sistemos gali veikti skirtingose serverinėse, kurios gali būti įvairiose geografinėse vietose \cite{shirriff2006method}.
		Toks sistemos išskirstymas pasižymi šiomis savybėmis:

		\begin{enumerate}
			\item{Pasiekiamumas (angl. {\it Availability}) -- vartotojas gali pasiekti sistemą bet kuriuo metu \cite{180327}.}
			\item{Patvarumas (angl. {\it Durability}) -- išskirstytos sistemos užtikrina duomenų išlaikymą ir pastovų veikimą net it tuo atveju, jeigu vienas iš paskirstytos sistemos mazgų nustotų veikti dėl sistemos sutrikimų, sukeltų programos klaidų arba gamtos katastrofų, tokių kaip: gaisrai, potvyniai ir panašios nelaimės. Ši savybė taip pat užtikrina sistemos gebėjimą atsistatyti ir neprarasti duomenų po minėtų įvykių \cite{5470366}.}
			\item{Plečiamumas (angl. {\it Scalability}) -- didėjant vartotojų skaičiui bei programų sistemos kompleksiškumui, išskirstytos sistemos užtikriną vertikalų (padidinti mazgo techninės įrangos galingumą) bei horizontalų (padidinti mazgų skaičių sistemoje) plečiamumą \cite{862209}.}
			\item{Našumas (angl. {\it Efficiency}) -- sistemos vartotojų skaičius dažniausiai būna nepastovus, jis kinta dienos metu arba atitinkamais metų periodais. Išskirstytos sistemos padeda užtikrinti našų įrangos resursų naudojimą sumažinant įrangos galingumą (kai vartotojų skaičius yra nedidelis) arba jį padidinant (kai sistemos apkrova padidėja).}
		\end{enumerate}

		Norint užtikrinti kuriamos sistemos kokybę dažnai yra rašomi testai. 
		Testai padeda atskleisti programos klaidas arba pasakyti ar naujas kodas nepaveikė seniau parašyto funkcionalumo \cite{819971}.
		Tačiau net ir laikantis gerųjų testavimo praktikų, nepavyksta išvengti programos klaidų.
		Net ir paskyrus daugiau resursų testavimui sudėtingose sistemose, pilnas sistemos testavimas yra neįmanomas \cite{sullivan2004software}.
		Norint rasti subtilesnius sutrikimus pačioje sistemos architektūroje, tokius kaip: dalinis mazgų neveikimas, lygiagrečių algoritmų klaidos naudojant keletą procesų, gedimų atsparumo ir atsistatymo po gedimų algoritmų klaidos bei rasti ir išspręsti kitus kraštutinius veikimo scenarijus \cite{newcombe2014use}, turime ieškoti kitų būdų. 
		Vienas iš jų formalios specifikacijos.

        Formalios specifikacijos(angl. {\it Formal specification}) yra matematinės technikos skirtos apibūdinti sistemų elgseną naudojant griežtas ir veiksmingas priemones \cite{holzmann1995improvement}.
        
		Turint sistemos formalią specifikaciją galima ja pasinaudoti vykdant formalų verifikavimą ir parodant, kad algoritmas yra adekvatus pagal sukurtą specifikaciją.
		Sudarinėti formalią sistemos specifikaciją galima ir nepradėjus įgyvendinti sistemos, turint tik jos architektūrą. 
  
		Formaliai verifikuota specifikacija suteikia informacijos apie architektūros korektiškumą ir įgalina objektyviai koreguoti sistemos architektūrą dar prieš pradedant ją įgyvendinti.
		Formalios specifikacijos sudaromos pasinaudojant tam tikromis kalbomis arba įrankiais.
		Viena iš tokių formalaus specifikavimo kalbų yra TLA\textsuperscript{+} \cite{lamport2002specifying}.
		

		TLA\textsuperscript{+} yra formalios specifikacijos kalba, kurią sukūrė Leslie Lamport \cite{lamport2002specifying}.
		Leslie Lamport 1980 metais sukūrė laiko veiksmų logiką (angl. {\it Temporal Logic of Actions}) \cite{10.1145/177492.177726}. Jis tai atliko pasinaudodamas Amir Pnueli 1977 metais sukurta laiko logika (angl. {\it Temporal Logic}) \cite{4567924}.
		1999 metais Leslie Lamport naudodamasis laiko veiksmų logika sukūrė formalaus specifikavimo kalbą TLA\textsuperscript{+} \cite{lamport2002specifying}.
		

        TLA\textsuperscript{+} kalba yra skirta kurti konkurencinių ir išskirstytų sistemų formalias specifikacijas ir jas verifikuoti.
		Naudojant TLA\textsuperscript{+} galima specifikuoti šias išskirstytų sistemų savybes \cite{lamport2019safety}:
  
		\begin{enumerate}
			\item{Gyvumas -- geri dalykai įvyksta programos vykdymo metu. Sistema galiausiai atliks jai paskirtą užduotį arba pasieks norimą būseną.}
			\item{Saugumas -- blogi dalykai neatsitiks programos vykdymo metu. Sistema nesustos veikti dėl netikėtai iškilusios klaidos.}
		\end{enumerate}
  
		Kadangi TLA\textsuperscript{+} specifikacijos yra rašomos formalia kalba, tai leidžia patikrinti sukurtos specifikacijos saugumo ir gyvumo savybes.

	Šias savybes mes galime patikrinti naudodamiesi TLC Model Checker (modelio tikrintoju).
		TLC yra išreikštinės būsenos (explicit-state) modelio tikrintojas, kurio paskirtis yra palaipsniui pasiekti visas galimas sistemos būsenas pagal nurodytą formalią specifikaciją \cite{yu1999model}.
		Tačiau kartais pagal sukurtą formalią specifikaciją susidaro labai daug būsenų, kurias sistema gali pasiekti, todėl naudoti TLC tampa nepraktiška.
		Tokiu atveju galime naudotis TLA\textsuperscript{+} specifikacijos įrodymo sistema TLAPS \cite{cousineau2012tla+}.
		TLAPS yra įrodymų sistema, skirta patikrinti TLA\textsuperscript{+} įrodymus.
		Šios sistemos paskirtis yra patikrinti pateiktus teoremų įrodymus.
		Įrodžius teoremą yra laikoma, kad TLA\textsuperscript{+} specifikacija yra korektiška.


		Yra ir kitų formalaus specifikavimo kalbų, kurias galėtume naudoti šiame darbe.
		Viena iš jų -- Z formalaus specifikavimo kalba \cite{O’Regan2017}, kuri sėkmingai buvo panaudota specifikuoti UNIX failų sistemą \cite{bowen1996formal} bei pasitelkta Oksfordo universiteto paskirstytų skaičiavimų projekte \cite{bowen1996formal}.
		Dar viena formalaus specifikavimo kalba yra VDA \cite{bjorner1978vienna}, kuria buvo specifikuoti bendros atminties sinchronizavimo algoritmai \cite{Slaats1998}.
		Tačiau šiam darbui buvo pasirinkta TLA\textsuperscript{+} kalba dėl jos pritaikymo išskirstytoms sistemoms naudojant būsenų mašiną \cite{lamport2002specifying}, esamų sėkmingo pritaikymo pavyzdžių specifikuojant išskirstytas sistemas \cite{newcombe2014use,kfkTla,rafttla,jiryu2020extreme} bei gausaus TLA\textsuperscript{+} įrankių pasirinkimo.

        Kuriant sistemos specifikaciją iškyla dvi pagrindinės problemos:
        
        \begin{enumerate}
            \item{Norint sukurti sistemos formalią specifikaciją reikia papildomo laiko ir žinių.}
            \item{Formali specifikacija gali neatitikti sistemos įgyvendinimo.}
        \end{enumerate}

        Šiuo metu tik nedidelė dalis programuotojų ir įmonių naudoja formalias specifikacijas savo
        programų kūrimo procese, todėl norint taikyti šią praktiką projekte, reikia apmokyti darbuotojus,
        o tai kainuoja bei užtrunka nemažai laiko.

        Tačiau net ir suprantant kaip kurti formalias specifikacijas, yra tikimybė padaryti aplaidumo klaidų, dėl kurių formali specifikacija nepilnai atitiktų tikrą sistemos įgyvendinimą.
    
        Kad įgyvendintas programinis kodas atitiktų formalią specifikaciją, mes galime testuoti kodą vieneto testais ir atlikinėti kodo peržiūras. Tačiau tai negarantuoja, kad klaidos, dėl specifikacijos
        neatitikimo, nepasieks vykdomosios aplinkos, sukeldamos problemų sistemos vartotojams.
        Skirtumą tarp formalios specifikacijos ir įgyvendinto kodo galime sumažinti generuojant kodą iš
        formalios specifikacijos arba generuojant specifikaciją iš programinio kodo. 

        Išskirstytas sistemas galima kurti naudojant bet kokią programavimo kalbą (ActiveMQ \cite{snyder2011activemq} naudoja Java programavimo kalbą, Apache Spark \cite{spark2018apache} naudoja Scala programavimo kalbą).
		Viena populiaresnių funkcinių programavimo kalbų naudojamų išskirstytų sistemų kūrime yra Elixir.
		Elixir yra dinaminė, funkcinė programavimo kalba skirta kurti išskirstytoms sistemoms\cite{juric2019elixir}.
		Elixir kalba veikia naudodama Erlang BEAM virtualią mašiną, kurios pagalba galima kurti mažo delsimo, išskirstytas, trikdžiams atsparias sistemas.
        
        Šiame darbe praplėsime anksčiau sukurtą metodą TLA\textsuperscript{+} specifikacijos generavimui iš nuoseklaus Elixir programinio kodo \cite{bravzenas2023tla+}. 
        Nuoseklus kodas yra Elixir programos pagrindas, tačiau norint apimti didesnę aibę išskirstytų
        Elixir sistemų, kad galėtume generuoti TLA\textsuperscript{+} formalias specifikacijas, turime sukurti generavimo metodus ir kitiems kalbos funkcionalumams.
        Vienas iš dažnai naudojamų Elixir kalbos funkcionalumų yra prižiūrintieji procesai(angl. {\it Supervizor process}).
        
        Elixir kalboje procesas yra pagrindinis sistemos elementas.
        Procesai yra lengvi, izoliuoti ir gali veikti lygiagrečiai.
        Tačiau norint užtikrinti procesų atsparumą gedimams (angl. {\it Fault tolerance}), mums reikalingas mechanizmas, kuriuo galėtume prižiūrėti procesus.
        Šį mechanizmą Elixir kalboje įgyvendina prižiūrintieji procesai.
        Prižiūrintysis procesas turi tris pagrindines paskirtis:
        \begin{enumerate}
            \item {Sukurti vaikinį procesą.}
            \item {Paleisti vaikinį procesą iš naujo, jei įvyksta vykdymo sutrikimas arba yra pasiekiama norima sąlyga.}
            \item {Išjungti vaikinį procesą programos veikimo pabaigoje.}
        \end{enumerate}

        Prižiūrintieji procesai gali valdyti kitus prižiūrinčiuosius procesus taip sudarydami vadinamąjį
        prižiūrėjimo medį(angl. {\it Supervision tree}).

        Šiame darbe bandysime sukurti metodą -- kaip iš Elixir kodo, naudojančio prižiūrėjimo procesus, išskirti TLA\textsuperscript{+} formalią specifikaciją.


	\sectionnonum{Temos aktualumas bei naujumas}
		Viena iš formalių specifikacijų ir TLA\textsuperscript{+} panaudojimo industrijoje sėkmės istorijų yra Amazon Web Service (AWS) komandos 2014 metais išleistas straipsnis \cite{newcombe2014use}.
		Straipsnyje rašoma, kad AWS komanda naudojo TLA\textsuperscript{+} sudarant formalias specifikacijas dešimtyje projektų. Tuo metu AWS turėjo 7 komandas, kurios naudojosi TLA\textsuperscript{+} kurdamos naujas programų sistemas.
		AWS sistemos specifikavimo metu buvo surasta 10 iki šiol neatrastų sisteminių klaidų, kurių atradimas ir pasiūlyti jų ištaisymai atskleidė tolimesnes sistemos klaidas, kurios taip pat buvo ištaisytos.
		Straipsnyje įvardinta ir kita, netiesioginė nauda gauta formaliai specifikuojant sistemas -- pagerėjęs bendras sistemos suvokimas, padidėjęs produktyvumas ir inovacijos.
		
		Dar viena sėkmės istorija yra 2018 metais Kafka Summit konferencijoje pristatyta Kafka TLA\textsuperscript{+} formali specifikacija, kurią sukūrė Jason Gustafson \cite{kfkTla}.
		Pristatyme buvo parodyta kaip pritaikant TLA\textsuperscript{+} bei specifikuojant Kafka duomenų replikavimo algoritmą buvo surastos ir pataisytos 3 retai atsitinkančios programos klaidos.
		Taisant taip pat rastos ir pataisytos dar kelios klaidos.

        Šie pavyzdžiai parodo, kad TLA\textsuperscript{+} specifikacijų naudojimas programų kūrimo procese padeda surasti klaidas ir padidinti sistemos atsparumą trikdžiams.

        Nors iš pateiktų pavyzdžių galima teigti, kad TLA\textsuperscript{+} specifikacijos suteikia naudos, tačiau sistemų kūrėjai vangiai naudoja jas programų kūrimo procese dėl aukštos mokymosi
        kartelės bei sukurto kodo atitikimo su formalia specifikacija.

        Matomas poreikis įrankiams ir metodams padedantiems palengvinti TLA\textsuperscript{+} specifikacijų kūrimo procesą bei užtikrinantiems geresnį specifikacijos atitikimą programiniam kodui. 

        Elixir kalba dažnai naudojama išskirstytoms sistemos \cite{expop}, kuriomis naudojasi daug vartotojų. 
        Tokioms sistemoms yra svarbu, kad jose būtų kuo mažiau klaidų ir sutrikimų.
        Trukdžiai tokiose sistemose gali paveikti daug vartotojų, sukelti saugumo spragas, paveikti sistemos kūrėjų reputaciją bei sukelti finansinių nuostolių.

        Šiame darbe sieksime sukurti taisyklių rinkinį, kurio pagalba Elixir sistemų formali specifikacija taptų lengvesnė bei labiau atitiktų įgyvendintą kodą.
        Darbe praplėsime ankščiau pradėtą TLA\textsuperscript{+} specifikacijos išskyrimą iš nuoseklaus Elixir kodo pridėdami taisyklių rinkinį, kuris išskirtų TLA\textsuperscript{+} iš Elixir prižiūrinčiųjų procesų kodo. 
        Šios taisyklės praplėstų aibę išskirstytų sistemų iš kurių kodo galėtume išskirti formalią specifikaciją, taip įgalindami Elixir programuotojus lengviau specifikuoti sistemas.

	\sectionnonum{Darbo tikslas}
		Sukurti metodą kuris išskiria TLA\textsuperscript{+} specifikaciją iš Elixir prižiūrinčiųjų procesų programinio kodo.

	
	\sectionnonum{Uždaviniai}
		\begin{enumerate}
			\item{Sukurti taisyklių rinkinį kuris išskirtų TLA\textsuperscript{+} specifikaciją iš Elixir prižiūrinčiųjų procesų.}
			\item{Pagal sudarytas taisykles įgyvendinti įrankį, kuris Elixir kodą verčia į TLA\textsuperscript{+} specifikaciją.}
			\item{Įvertinti sugeneruotos specifikacijos teisingumą.}
			\item{Surasti atviro kodo sistemų naudojančių Elixir prižiūrinčiuosius procesus ir patikrinti to kodo teisingumą sugeneruojant TLA\textsuperscript{+} specifikaciją.}
		\end{enumerate}
	
	\sectionnonum{Laukiami rezultatai}
		\begin{enumerate}
			\item{Elixir prižiūrinčiųjų procesų kodo pavertimas į TLA\textsuperscript{+} specifikaciją.}
			\item{Įrodytas Elixir kodo pavertimo į TLA\textsuperscript{+} specifikaciją adekvatumas.}
			\item{Elixir kodo ir TLA\textsuperscript{+} specifikacijos sutapimo įvertinimas.}
			\item{Papildytas TLA\textsuperscript{+} specifikacijos iš Elixir kodo generatorius.}
		\end{enumerate}

	\sectionnonum{Hipotezės}
		\begin{itemize}
			\item H0 (nulinė hipotezė) -- metodas paverčiantis Elixir prižiūrinčiųjų procesų kodą į TLA\textsuperscript{+} specifikaciją neegzistuoja.
			\item H1 (alternatyvi hipotezė) -- metodas paverčiantis Elixir prižiūrinčiųjų procesų kodą į TLA\textsuperscript{+} specifikaciją egzistuoja.
		\end{itemize}
		
	\pagebreak
	\printbibliography[heading=bibintoc] 

\end{document}
