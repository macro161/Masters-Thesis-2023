\documentclass{VUMIFPSmagistrinis}
\usepackage{algorithmicx}
\usepackage{algorithm}
\usepackage{algpseudocode}
\usepackage{amsfonts}
\usepackage{amsmath}
\usepackage{bm}
\usepackage{caption}
\usepackage{color}
\usepackage{float}
\usepackage{graphicx}
\usepackage{listings}
\usepackage{subfig}
\usepackage{wrapfig}



\university{Vilniaus universitetas}
\faculty{Matematikos ir informatikos fakultetas}
\department{Informatikos institutas}
\papertype{Magistro darbo literatūros pažvalga}
\title{Formalių specifikacijų taikymas projektuojant išskirstytas sistemas}
\titleineng{Applying Formal Specifications to Design Distributed Systems}
\author{Matas Savickis}

\supervisor{Karolis Petrauskas, Doc., Dr.}
\reviewer{Valaitis Vytautas, Asist., Dr.}
\date{Vilnius – \the\year}

\bibliography{bibliografija}

\begin{document}
\pagenumbering{arabic}

\maketitle

\tableofcontents

\sectionnonum{Įvadas}
Šiame skyriuje aptarsime su darbu susijusia literatūrą ir kokie tyrimai buvo atlikti iki šio darbo bei suformuosime problemą kurią planuojame spręsti šiame darbe.


\section{Raft protokolas}
	\subsection{Protokolas}
	\subsection{Raft TLA+}

\section{Kafka Raft}
	\subsection{Motyvacija}
	\subsection{Kafta Raft}
	\subsection{Raft vs Kafka Raft}




\printbibliography[heading=bibintoc] 

\end{document}
